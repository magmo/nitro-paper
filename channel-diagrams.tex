% https://tex.stackexchange.com/a/55604

\usepackage{tikz}
\usetikzlibrary{shapes,backgrounds,calc}
\usetikzlibrary{positioning, shapes.geometric}

\makeatletter
\tikzset{fill halves/.style  args={#1,#2}{%
  circle,
  postaction={%
    insert path={
      \pgfextra{% 
        % This entire script assumes that we're working with a circle, making use of the anchors
        % we expect to find on a circle
        % Calculates "insiderad" by looking at the distance from the center to the east anchor
        \pgfpointdiff{\pgfpointanchor{\pgf@node@name}{center}}%
                    {\pgfpointanchor{\pgf@node@name}{east}}%            
        \pgfmathsetmacro\insiderad{\pgf@x}
        % We start at the east anchor of the node and then move in by "pgflinewidth"
        % then we draw an arc from 0 to 180
        \fill[#1] (\pgf@node@name.base) ([xshift=-\pgflinewidth/sqrt(2), yshift=-\pgflinewidth/sqrt(2)]\pgf@node@name.north east) arc
                          (45:225:\insiderad-\pgflinewidth)--cycle;
        \fill[#2] (\pgf@node@name.base) ([xshift=\pgflinewidth/sqrt(2), yshift=\pgflinewidth/sqrt(2)]\pgf@node@name.south west)  arc
                            (225:405:\insiderad-\pgflinewidth)--cycle;
        \draw[-] (\pgf@node@name.north east) to (\pgf@node@name.south west);
      }
    }
  }
}
}  

\tikzset{fill thirds/.style  args={#1,#2,#3}{%
  circle,
  postaction={%
    insert path={
      \pgfextra{% 
        % This entire script assumes that we're working with a circle, making use of the anchors
        % we expect to find on a circle
        % Calculates "insiderad" by looking at the distance from the center to the east anchor
        \pgfpointdiff{\pgfpointanchor{\pgf@node@name}{center}}%
                    {\pgfpointanchor{\pgf@node@name}{east}}%            
        \pgfmathsetmacro\insiderad{\pgf@x - \pgflinewidth}
        % We start at the east anchor of the node and then move in by "pgflinewidth"
        % then we draw an arc from 0 to 180
        \fill[#1] (\pgf@node@name.center) -- ++ (90:\insiderad pt) arc (90:210:\insiderad pt) --cycle;
        \fill[#2] (\pgf@node@name.center) -- ++ (210:\insiderad pt) arc (210:330:\insiderad pt) --cycle;
        \fill[#3] (\pgf@node@name.center) -- ++ (330:\insiderad pt) arc (330:450:\insiderad pt) --cycle;
        \draw (\pgf@node@name.center) -- ++ (90:\insiderad pt);
        \draw (\pgf@node@name.center) -- ++ (210:\insiderad pt);
        \draw (\pgf@node@name.center) -- ++ (330:\insiderad pt);
      }
    }
  }
}
}  

\tikzset{semithirds/.style  args={#1,#2,#3}{%
  semicircle,
  rotate=180,
  postaction={%
    insert path={
      \pgfextra{% 
        % This entire script assumes that we're working with a circle, making use of the anchors
        % we expect to find on a circle
        % Calculates "insiderad" by looking at the distance from the center to the east anchor
        \pgfpointdiff{\pgfpointanchor{\pgf@node@name}{south}}%
                    {\pgfpointanchor{\pgf@node@name}{north}}%            
        \pgfmathsetmacro\insiderad{\pgf@y-1.5\pgflinewidth}
        % We start at the east anchor of the node and then move in by "pgflinewidth"
        % then we draw an arc from 0 to 180
        \fill[#1] ([yshift=\pgflinewidth/2]\pgf@node@name.south) -- ++ (0:\insiderad pt) arc (0:60:\insiderad pt) --cycle;
        \fill[#2] ([yshift=\pgflinewidth/2]\pgf@node@name.south) -- ++ (60:\insiderad pt) arc (60:120:\insiderad pt) --cycle;
        \fill[#3] ([yshift=\pgflinewidth/2]\pgf@node@name.south) -- ++ (120:\insiderad pt) arc (120:180:\insiderad pt) --cycle;
        % \draw ([yshift=\pgflinewidth/2]\pgf@node@name.south) -- ++ (0:\insiderad pt);
        \draw ([yshift=\pgflinewidth/2]\pgf@node@name.south) -- ++ (60:\insiderad pt);
        \draw ([yshift=\pgflinewidth/2]\pgf@node@name.south) -- ++ (120:\insiderad pt);
        % \draw ([yshift=\pgflinewidth/2]\pgf@node@name.south) -- ++ (180:\insiderad pt);
      }
    }
  }
}
}  

\tikzset{
  guarantee/.pic={
  \pgfmathsetmacro{\angle}{45};
  \pgfmathsetmacro{\inradius}{2};
  \pgfmathsetmacro{\separation}{1.5};
  \pgfmathsetmacro{\clearance}{3};

  \coordinate (-incenter) at (0, 0);
  \path (-incenter) -- +({\angle}:\inradius mm) coordinate (-rightp2);
  \path (-incenter) -- +({180-\angle}:\inradius mm) coordinate (-leftp2);
  \path (-incenter) -- +(270:\inradius mm) coordinate (-basep2);

  \path (-rightp2) -- + ({90+\angle}:\separation mm) coordinate (-rightp1);
  \path (-rightp2) -- + ({90+\angle}:-\separation mm) coordinate (-rightp3);

  \path (-leftp2) -- + ({90-\angle}:\separation mm) coordinate (-leftp3);
  \path (-leftp2) -- + ({90-\angle}:-\separation mm) coordinate (-leftp1);

  \path (-basep2) -- +(0:\separation mm) coordinate (-basep3);
  \path (-basep2) -- +(0:-\separation mm) coordinate (-basep1);

  \begin{scope}[-, every to/.style={out={180+\angle}, in=90}]
    \draw (-rightp1) to (-basep1);
    \draw (-rightp2) to (-basep2);
    \draw (-rightp3) to (-basep3);
  \end{scope}

  \begin{scope}[-, every to/.style={out={-\angle}, in=90}]
    \draw (-leftp1) to (-basep1);
    \draw (-leftp2) to (-basep2);
    \draw (-leftp3) to (-basep3);
  \end{scope}

  \path (-rightp1) -- +(\angle:\clearance mm) coordinate (-Rightp1);
  \path (-rightp2) -- +(\angle:\clearance mm) coordinate (-Rightp2);
  \path (-rightp3) -- +(\angle:\clearance mm) coordinate (-Rightp3);

  \path (-leftp1) -- +({180-\angle}:\clearance mm) coordinate (-Leftp1);
  \path (-leftp2) -- +({180-\angle}:\clearance mm) coordinate (-Leftp2);
  \path (-leftp3) -- +({180-\angle}:\clearance mm) coordinate (-Leftp3);

  \begin{scope}[-, every to/.style={out={-\angle}, in={180-\angle}}]
    \draw (-Leftp1) to (-leftp1);
    \draw (-Leftp2) to (-leftp3);
    \draw (-Leftp3) to (-leftp2);
  \end{scope}

  \begin{scope}[-, every to/.style={out={180+\angle}, in={\angle}}]
    \draw (-Rightp1) to (-rightp2);
    \draw (-Rightp2) to (-rightp3);
    \draw (-Rightp3) to (-rightp1);
  \end{scope}

  \node[semicircle, draw, rotate=45, anchor=south] (-G0) at (-Leftp2) {};
  \node[semicircle, draw, rotate=-45, anchor=south] (-G1) at (-Rightp2) {};
  \node[semicircle, draw, rotate=180, anchor=south] (-J) at (-basep2) {};
  }
}


\tikzset{
  guarantee1/.pic={
  \pgfmathsetmacro{\angle}{45};
  \pgfmathsetmacro{\inradius}{2};
  \pgfmathsetmacro{\separation}{1.5};
  \pgfmathsetmacro{\clearance}{3};

  \coordinate (-incenter) at (0, 0);
  \path (-incenter) -- +({\angle}:\inradius mm) coordinate (-rightp2);
  \path (-incenter) -- +({180-\angle}:\inradius mm) coordinate (-leftp2);
  \path (-incenter) -- +(270:\inradius mm) coordinate (-basep2);

  \path (-rightp2) -- + ({90+\angle}:\separation mm) coordinate (-rightp1);
  \path (-rightp2) -- + ({90+\angle}:-\separation mm) coordinate (-rightp3);

  \path (-leftp2) -- + ({90-\angle}:\separation mm) coordinate (-leftp3);
  \path (-leftp2) -- + ({90-\angle}:-\separation mm) coordinate (-leftp1);

  \path (-basep2) -- +(0:\separation mm) coordinate (-basep3);
  \path (-basep2) -- +(0:-\separation mm) coordinate (-basep1);

  \begin{scope}[-, every to/.style={out={180+\angle}, in=90}]
    \draw (-rightp2) to (-basep2);
  \end{scope}

  \begin{scope}[-, every to/.style={out={-\angle}, in=90}]
    \draw (-leftp2) to (-basep2);
  \end{scope}

  \path (-rightp2) -- +(\angle:\clearance mm) coordinate (-Rightp2);

  \path (-leftp2) -- +({180-\angle}:\clearance mm) coordinate (-Leftp2);

  \begin{scope}[-, every to/.style={out={-\angle}, in={180-\angle}}]
    \draw (-Leftp2) to (-leftp2);
  \end{scope}

  \begin{scope}[-, every to/.style={out={180+\angle}, in={\angle}}]
    \draw (-Rightp2) to (-rightp2);
  \end{scope}

  \node[semicircle, draw, rotate=45, anchor=south] (-G0) at (-Leftp2) {};
  \node[semicircle, draw, rotate=-45, anchor=south] (-G1) at (-Rightp2) {};
  \node[semicircle, draw, rotate=180, anchor=south] (-J) at (-basep2) {};
  }
}


\tikzset{
  guarantee3bad/.pic={
  \pgfmathsetmacro{\angle}{45};
  \pgfmathsetmacro{\inradius}{2};
  \pgfmathsetmacro{\separation}{1.5};
  \pgfmathsetmacro{\clearance}{3};

  \coordinate (-incenter) at (0, 0);
  \path (-incenter) -- +({\angle}:\inradius mm) coordinate (-rightp2);
  \path (-incenter) -- +({180-\angle}:\inradius mm) coordinate (-leftp2);
  \path (-incenter) -- +(270:\inradius mm) coordinate (-basep2);

  \path (-rightp2) -- + ({90+\angle}:\separation mm) coordinate (-rightp1);
  \path (-rightp2) -- + ({90+\angle}:-\separation mm) coordinate (-rightp3);

  \path (-leftp2) -- + ({90-\angle}:\separation mm) coordinate (-leftp3);
  \path (-leftp2) -- + ({90-\angle}:-\separation mm) coordinate (-leftp1);

  \path (-basep2) -- +(0:\separation mm) coordinate (-basep3);
  \path (-basep2) -- +(0:-\separation mm) coordinate (-basep1);

  \begin{scope}[-, every to/.style={out={180+\angle}, in=90}]
    \draw (-rightp1) to (-basep1);
    \draw (-rightp2) to (-basep2);
    \draw (-rightp3) to (-basep3);
  \end{scope}

  \begin{scope}[-, every to/.style={out={-\angle}, in=90}]
    \draw (-leftp1) to (-basep1);
    \draw (-leftp2) to (-basep2);
    \draw (-leftp3) to (-basep3);
  \end{scope}

  \path (-rightp1) -- +(\angle:\clearance mm) coordinate (-Rightp1);
  \path (-rightp2) -- +(\angle:\clearance mm) coordinate (-Rightp2);
  \path (-rightp3) -- +(\angle:\clearance mm) coordinate (-Rightp3);

  \path (-leftp1) -- +({180-\angle}:\clearance mm) coordinate (-Leftp1);
  \path (-leftp2) -- +({180-\angle}:\clearance mm) coordinate (-Leftp2);
  \path (-leftp3) -- +({180-\angle}:\clearance mm) coordinate (-Leftp3);

  \begin{scope}[-, every to/.style={out={-\angle}, in={180-\angle}}]
    \draw (-Leftp1) to (-leftp1);
    \draw (-Leftp2) to (-leftp2);
    \draw (-Leftp3) to (-leftp3);
  \end{scope}

  \begin{scope}[-, every to/.style={out={180+\angle}, in={\angle}}]
    \draw (-Rightp1) to (-rightp1);
    \draw (-Rightp2) to (-rightp3);
    \draw (-Rightp3) to (-rightp2);
  \end{scope}

  \node[semicircle, draw, rotate=45, anchor=south] (-G0) at (-Leftp2) {};
  \node[semicircle, draw, rotate=-45, anchor=south] (-G1) at (-Rightp2) {};
  \node[semicircle, draw, rotate=180, anchor=south] (-J) at (-basep2) {};
  }
}

\tikzset{
  guarantee2/.pic={
    % to get a two pin guarantee, we'll reduce the distance of the three pin and ignore the middle pin
  \pgfmathsetmacro{\angle}{45};
  \pgfmathsetmacro{\inradius}{2};
  \pgfmathsetmacro{\separation}{1};
  \pgfmathsetmacro{\clearance}{3};

  \coordinate (-incenter) at (0, 0);
  \path (-incenter) -- +({\angle}:\inradius mm) coordinate (-rightp2);
  \path (-incenter) -- +({180-\angle}:\inradius mm) coordinate (-leftp2);
  \path (-incenter) -- +(270:\inradius mm) coordinate (-basep2);

  \path (-rightp2) -- + ({90+\angle}:\separation mm) coordinate (-rightp1);
  \path (-rightp2) -- + ({90+\angle}:-\separation mm) coordinate (-rightp3);

  \path (-leftp2) -- + ({90-\angle}:\separation mm) coordinate (-leftp3);
  \path (-leftp2) -- + ({90-\angle}:-\separation mm) coordinate (-leftp1);

  \path (-basep2) -- +(0:\separation mm) coordinate (-basep3);
  \path (-basep2) -- +(0:-\separation mm) coordinate (-basep1);

  \begin{scope}[-, every to/.style={out={180+\angle}, in=90}]
    \draw (-rightp1) to (-basep1);
    \draw (-rightp3) to (-basep3);
  \end{scope}

  \begin{scope}[-, every to/.style={out={-\angle}, in=90}]
    \draw (-leftp1) to (-basep1);
    \draw (-leftp3) to (-basep3);
  \end{scope}

  \path (-rightp1) -- +(\angle:\clearance mm) coordinate (-Rightp1);
  \path (-rightp2) -- +(\angle:\clearance mm) coordinate (-Rightp2);
  \path (-rightp3) -- +(\angle:\clearance mm) coordinate (-Rightp3);

  \path (-leftp1) -- +({180-\angle}:\clearance mm) coordinate (-Leftp1);
  \path (-leftp2) -- +({180-\angle}:\clearance mm) coordinate (-Leftp2);
  \path (-leftp3) -- +({180-\angle}:\clearance mm) coordinate (-Leftp3);

  \begin{scope}[-, every to/.style={out={-\angle}, in={180-\angle}}]
    \draw (-Leftp1) to (-leftp1);
    \draw (-Leftp3) to (-leftp3);
  \end{scope}

  \begin{scope}[-, every to/.style={out={180+\angle}, in={\angle}}]
    \draw (-Rightp1) to (-rightp1);
    \draw (-Rightp3) to (-rightp3);
  \end{scope}

  \node[semicircle, draw, rotate=45, anchor=south] (-G0) at (-Leftp2) {};
  \node[semicircle, draw, rotate=-45, anchor=south] (-G1) at (-Rightp2) {};
  \node[semicircle, draw, rotate=180, anchor=south] (-J) at (-basep2) {};
  }
}


\tikzset{
  guarantee2bad/.pic={
    % to get a two pin guarantee, we'll reduce the distance of the three pin and ignore the middle pin
  \pgfmathsetmacro{\angle}{45};
  \pgfmathsetmacro{\inradius}{2};
  \pgfmathsetmacro{\separation}{1};
  \pgfmathsetmacro{\clearance}{3};

  \coordinate (-incenter) at (0, 0);
  \path (-incenter) -- +({\angle}:\inradius mm) coordinate (-rightp2);
  \path (-incenter) -- +({180-\angle}:\inradius mm) coordinate (-leftp2);
  \path (-incenter) -- +(270:\inradius mm) coordinate (-basep2);

  \path (-rightp2) -- + ({90+\angle}:\separation mm) coordinate (-rightp1);
  \path (-rightp2) -- + ({90+\angle}:-\separation mm) coordinate (-rightp3);

  \path (-leftp2) -- + ({90-\angle}:\separation mm) coordinate (-leftp3);
  \path (-leftp2) -- + ({90-\angle}:-\separation mm) coordinate (-leftp1);

  \path (-basep2) -- +(0:\separation mm) coordinate (-basep3);
  \path (-basep2) -- +(0:-\separation mm) coordinate (-basep1);

  \begin{scope}[-, every to/.style={out={180+\angle}, in=90}]
    \draw (-rightp1) to (-basep1);
    \draw (-rightp3) to (-basep3);
  \end{scope}

  \begin{scope}[-, every to/.style={out={-\angle}, in=90}]
    \draw (-leftp1) to (-basep1);
    \draw (-leftp3) to (-basep3);
  \end{scope}

  \path (-rightp1) -- +(\angle:\clearance mm) coordinate (-Rightp1);
  \path (-rightp2) -- +(\angle:\clearance mm) coordinate (-Rightp2);
  \path (-rightp3) -- +(\angle:\clearance mm) coordinate (-Rightp3);

  \path (-leftp1) -- +({180-\angle}:\clearance mm) coordinate (-Leftp1);
  \path (-leftp2) -- +({180-\angle}:\clearance mm) coordinate (-Leftp2);
  \path (-leftp3) -- +({180-\angle}:\clearance mm) coordinate (-Leftp3);

  \begin{scope}[-, every to/.style={out={-\angle}, in={180-\angle}}]
    \draw (-Leftp1) to (-leftp1);
    \draw (-Leftp3) to (-leftp3);
  \end{scope}

  \begin{scope}[-, every to/.style={out={180+\angle}, in={\angle}}]
    \draw (-Rightp1) to (-rightp3);
    \draw (-Rightp3) to (-rightp1);
  \end{scope}

  \node[semicircle, draw, rotate=45, anchor=south] (-G0) at (-Leftp2) {};
  \node[semicircle, draw, rotate=-45, anchor=south] (-G1) at (-Rightp2) {};
  \node[semicircle, draw, rotate=180, anchor=south] (-J) at (-basep2) {};
  }
}

\tikzset{
  guarantee2other/.pic={
    % to get a two pin guarantee, we'll reduce the distance of the three pin and ignore the middle pin
  \pgfmathsetmacro{\angle}{45};
  \pgfmathsetmacro{\inradius}{2};
  \pgfmathsetmacro{\separation}{1};
  \pgfmathsetmacro{\clearance}{3};

  \coordinate (-incenter) at (0, 0);
  \path (-incenter) -- +({\angle}:\inradius mm) coordinate (-rightp2);
  \path (-incenter) -- +({180-\angle}:\inradius mm) coordinate (-leftp2);
  \path (-incenter) -- +(270:\inradius mm) coordinate (-basep2);

  \path (-rightp2) -- + ({90+\angle}:\separation mm) coordinate (-rightp1);
  \path (-rightp2) -- + ({90+\angle}:-\separation mm) coordinate (-rightp3);

  \path (-leftp2) -- + ({90-\angle}:\separation mm) coordinate (-leftp3);
  \path (-leftp2) -- + ({90-\angle}:-\separation mm) coordinate (-leftp1);

  \path (-basep2) -- +(0:\separation mm) coordinate (-basep3);
  \path (-basep2) -- +(0:-\separation mm) coordinate (-basep1);

  \begin{scope}[-, every to/.style={out={180+\angle}, in=90}]
    \draw (-rightp1) to (-basep1);
    \draw (-rightp3) to (-basep3);
  \end{scope}

  \begin{scope}[-, every to/.style={out={-\angle}, in=90}]
    \draw (-leftp1) to (-basep1);
    \draw (-leftp3) to (-basep3);
  \end{scope}

  \path (-rightp1) -- +(\angle:\clearance mm) coordinate (-Rightp1);
  \path (-rightp2) -- +(\angle:\clearance mm) coordinate (-Rightp2);
  \path (-rightp3) -- +(\angle:\clearance mm) coordinate (-Rightp3);

  \path (-leftp1) -- +({180-\angle}:\clearance mm) coordinate (-Leftp1);
  \path (-leftp2) -- +({180-\angle}:\clearance mm) coordinate (-Leftp2);
  \path (-leftp3) -- +({180-\angle}:\clearance mm) coordinate (-Leftp3);

  \begin{scope}[-, every to/.style={out={-\angle}, in={180-\angle}}]
    \draw (-Leftp1) to (-leftp3);
    \draw (-Leftp3) to (-leftp1);
  \end{scope}

  \begin{scope}[-, every to/.style={out={180+\angle}, in={\angle}}]
    \draw (-Rightp1) to (-rightp1);
    \draw (-Rightp3) to (-rightp3);
  \end{scope}

  \node[semicircle, draw, rotate=45, anchor=south] (-G0) at (-Leftp2) {};
  \node[semicircle, draw, rotate=-45, anchor=south] (-G1) at (-Rightp2) {};
  \node[semicircle, draw, rotate=180, anchor=south] (-J) at (-basep2) {};
  }
}




\tikzset{
  circA/.style={ draw, circle, fill=red!50, minimum size=4mm }
}

\tikzset{
  circB/.style={ draw, circle, fill=blue!50, minimum size=4mm  }
}

\tikzset{
  circI/.style={ draw, circle, fill=orange!50, minimum size=4mm  }
}

\tikzset{
  circEmpty/.style={ draw, circle, dotted, minimum size=4mm  }
}

\tikzset{
  circAB/.style={ draw, circle, fill halves={red!50,blue!50}, minimum size=4mm }
}
\tikzset{
  circAI/.style={ draw, circle, fill halves={red!50,orange!50}, minimum size=4mm }
}
\tikzset{
  circBI/.style={ draw, circle, fill halves={blue!50,orange!50}, minimum size=4mm }
}
\tikzset{
  circABI/.style={ draw, semicircle, rotate=180, minimum size=2mm }
}
\tikzset{
  guarAI/.style={ draw, semicircle, minimum size=2mm }
}
\tikzset{
  guarBI/.style={ draw, semicircle, minimum size=2mm }
}

\tikzset{
  sqadj/.style={ draw, minimum size=4mm }
}

\tikzset{
  valuebox/.style={draw, rectangle split, rectangle split horizontal, rectangle split parts=2}
}

\makeatother  
