
\usepackage{amsmath}
\usepackage{amsthm}
\usepackage{amsthm}
\usepackage{eucal}
\usepackage{amssymb}
\usepackage{color}
\usepackage{tikz}
\usetikzlibrary{shapes,arrows.meta,backgrounds}
\usepackage{etoolbox}
\usepackage{url} % for urls in bibliography
\usepackage[parfill]{parskip} % no paragraph indents, leave blank line
\usepackage{graphicx}
\usepackage{subcaption}
\usepackage{listings}
\usepackage{xifthen}% provides \isempty test
\usepackage{todonotes}

\graphicspath{ {./drawio/} }

% Need this to keep the space before theorems when using parfill parskip
% https://tex.stackexchange.com/questions/25346/wrong-spacing-before-theorem-environment-amsthm
\begingroup
    \makeatletter
    \@for\theoremstyle:=definition,remark,plain\do{%
        \expandafter\g@addto@macro\csname th@\theoremstyle\endcsname{%
            \addtolength\thm@preskip\parskip
            }%
        }
\endgroup

\DeclareRobustCommand{\rchi}{{\mathpalette\irchi\relax}}
\newcommand{\irchi}[2]{\raisebox{\depth}{$#1\chi$}} % inner command, used by \rchi

\usepackage{mathtools}
\usepackage{bm}
\usepackage{stmaryrd} % for llbracket and rrbracket

\theoremstyle{definition}
\newtheorem{example}{Example}[section]
\newtheorem{defn}{Definition}[section]
\newtheorem{theorem}{Theorem}[section]
\newtheorem{conjecture}{Conjecture}[section]

\newcommand{\adj}[1]{\llbracket #1 \rrbracket} 
\newcommand{\enf}[1]{[#1]} 

\newcommand{\holds}[3]{#1 %
  \ifthenelse{\isempty{#2}}{}{: #2} %
  \ifthenelse{\isempty{#3}}{}{\mapsto #3} %
} 
% \newcommand{\alloc}[1]{( #1 )} 
% \newcommand{\guar}[2]{( #1 | #2 )} 

\newcommand{\finalizable}[3]{[#1 \mapsto #2]_{#3}}
\newcommand{\transfer}[2]{\mbox{Transfer}(#1, #2)}
\newcommand{\allocation}[2]{\mbox{Alloc}(#1, #2)}
\newcommand{\guarantee}[3]{\mbox{Guar}(#1, #2, [#3])}
\newcommand{\claim}[2]{Claim(#1, #2)}

\newcommand{\version}{\mbox{version}}
\newcommand{\outcome}{\mbox{outcome}}
\newcommand{\holdings}{\mbox{holdings}}
\newcommand{\hash}{\mbox{hash}}
\newcommand{\isFinal}{\mbox{isFinal}}
\newcommand{\nonce}{\mbox{nonce}}
\newcommand{\peers}{\mbox{peers}}
\newcommand{\appDefinition}{\mbox{appDef}}
\newcommand{\appData}{\mbox{appData}}


\newcommand{\finalizationTime}{\mbox{finalizationTime}}
\newcommand{\challengeDuration}{\mbox{challengeDuration}}
\newcommand{\latestSupportedState}{\mbox{latestSupportedState}}

\newcommand{\fixedPart}{\mbox{fixedPart}}
\newcommand{\true}{\mbox{true}}
\newcommand{\statusOf}{\mbox{statusOf}}
\newcommand{\now}{\mbox{now}}
\newcommand{\amount}{\mbox{amount}}

\newcommand{\timeToPayment}{time to payment}